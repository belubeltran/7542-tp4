\documentclass{article}

%% PAQUETES

% Paquetes generales
\usepackage[margin=2cm, paperwidth=210mm, paperheight=297mm]{geometry}
\usepackage[spanish]{babel}
\usepackage[utf8]{inputenc}
\usepackage{gensymb}

% Paquetes para estilos
\usepackage{textcomp}
\usepackage{setspace}
\usepackage{colortbl}
\usepackage{color}
\usepackage{color}
\usepackage{upquote}
\usepackage{xcolor}
\usepackage{listings}
\usepackage{caption}
\usepackage[T1]{fontenc}
\usepackage[scaled]{beramono}

% Paquetes extras
\usepackage{amssymb}
\usepackage{float}
\usepackage{graphicx}
\usepackage{url}

%% Fin PAQUETES


% Definición de preferencias para la impresión de código fuente.
%% Colores
\definecolor{gray99}{gray}{.99}
\definecolor{gray95}{gray}{.95}
\definecolor{gray75}{gray}{.75}
\definecolor{gray50}{gray}{.50}
\definecolor{keywords_blue}{rgb}{0.13,0.13,1}
\definecolor{comments_green}{rgb}{0,0.5,0}
\definecolor{strings_red}{rgb}{0.9,0,0}

%% Caja de código
\DeclareCaptionFont{white}{\color{white}}
\DeclareCaptionFont{style_labelfont}{\color{black}\textbf}
\DeclareCaptionFont{style_textfont}{\it\color{black}}
\DeclareCaptionFormat{listing}{\colorbox{gray95}{\parbox{16.78cm}{#1#2#3}}}
\captionsetup[lstlisting]{format=listing,labelfont=style_labelfont,textfont=style_textfont}

\lstset{
	aboveskip = {1.5\baselineskip},
	backgroundcolor = \color{gray99},
	basicstyle = \ttfamily\footnotesize,
	breakatwhitespace = true,   
	breaklines = true,
	captionpos = t,
	columns = fixed,
	commentstyle = \color{comments_green},
	escapeinside = {\%*}{*)}, 
	extendedchars = true,
	frame = lines,
	keywordstyle = \color{keywords_blue}\bfseries,
	language = Oz,                       
	numbers = left,
	numbersep = 5pt,
	numberstyle = \tiny\ttfamily\color{gray50},
	prebreak = \raisebox{0ex}[0ex][0ex]{\ensuremath{\hookleftarrow}},
	rulecolor = \color{gray75},
	showspaces = false,
	showstringspaces = false, 
	showtabs = false,
	stepnumber = 1,
	stringstyle = \color{strings_red},                                    
	tabsize = 2,
	title = \null, % Default value: title=\lstname
	upquote = true,                  
}

%% FIGURAS
\captionsetup[figure]{labelfont=bf,textfont=it}
%% TABLAS
\captionsetup[table]{labelfont=bf,textfont=it}

% COMANDOS

%% Titulo de las cajas de código
\renewcommand{\lstlistingname}{Código}
%% Titulo de las figuras
\renewcommand{\figurename}{Figura}
%% Titulo de las tablas
\renewcommand{\tablename}{Tabla}
%% Referencia a los códigos
\newcommand{\refcode}[1]{\textit{Código \ref{#1}}}
%% Referencia a las imagenes
\newcommand{\refimage}[1]{\textit{Imagen \ref{#1}}}


\begin{document}

% Inserción del título, autores y fecha.
\title{\Large 75.42 Taller de Programación I \\ 
	  \medskip\Huge Informe: Ejercicio N° 4  \\
	  \bigskip\Large\textit{``El código Draka (desencripción por fuerza bruta)''}}
\date{}
\maketitle




% INTRODUCCIÓN
\section{Introducción}
	
	Nuestros agentes han finalmente recuperado el código utilizado por los \textit{Draka} para encriptar y desencriptar sus mensajes. Si bien el código es relativamente simple, no ha sido posible encontrarle vulnerabilidades hasta el momento. Solo se sabe que los textos encriptados son ASCII \cite{ASCII} (códigos de carácter menores a 128) y que las claves solo utilizan dígitos ASCII. Aparentemente, debido a esto, la única opción para recuperar el texto será realizar un ataque de \textit{fuerza bruta} \cite{FB}.
	\par
	Las claves utilizadas son de longitud suficiente como para hacer que un ataque por fuerza bruta con una sola máquina lleve una excesiva cantidad de tiempo. Por lo tanto, se optará por una solución distribuida siguiendo un esquema cliente-servidor: el servidor se encargará de fraccionar el trabajo y los clientes de resolver cada una de estas partes, enviando los resultados al servidor.
	\par
	Detalles mas precisos de la problemática y de las condiciones preestablecidas se pueden encontrar en el enunciado del ejercicio\footnote{Se ha evitado hacer un relevamiento de la totalidad de la información que nos fue conferida, de manera de poder mantener el foco del informe en la forma en que se ha encarado la solución del problema.}.
\bigskip




% CONSIDERACIONES DE DISEÑO
\section{Consideraciones de diseño}

	Para la correcta implementación de la solución fue necesario plantear y establecer cómo se debería comportar el sistema ante ciertas situaciones que no fueron especificadas en el enunciado del problema. A continuación se listan las contemplaciones instauradas:

\begin{itemize}
	\itemsep=3pt \topsep=0pt \partopsep=0pt \parskip=0pt \parsep=0pt

	\item Al producirse una solicitud de trabajo por parte de un cliente, si el servidor le indica que no hay trabajo para asignarle, ambos entes cerraran la conexión que los vinculaba;

	\item El servidor almacenará todas las posibles claves, para una vez finalizada la actividad de todos los clientes, determinar si existe una ambig\"uedad o si es única la clave;

	\item En los comandos ingresados a través de la terminal del servidor, se considera que el caractér de fin de línea, ``\textit{$\backslash$n}'', equivale a presionar enter una vez tipeado el comando deseado;

	\item En el envío y en la recepción de datos se considera como fin de mensaje al caractér ``\textit{$\backslash$n}'' establecido por el protocolo.

	\item Al provocar la salida forzada del lado del servidor mediante el ingreso de ``q$\backslash$n'' por entrada estándar, previo a la finalización se enviará a la salida estándar el estado del servidor en el momento en que se invocó su cierre. Es muy probable que al ser forzada la salida, el estado a mostrar sea \textit{Not finished}, ya que de otra manera, el servidor habría finalizado automaticamente su ejecución.
\end{itemize}
\medskip




% DISEÑO
\section{Diseño}

	Como creemos que la escalabilidad de un sistema (en este caso del programa) es muy importante, es que se ha decidido modularizar consistentemente ciertas partes que conforman las especificaciones, de manera de permitirnos, en un futuro, agregar o mejorar más fácilmente el funcionamiento del mismo. Para lograr este propósito se hará uso, entre otras cosas, de TDAs y estructuras.
	\par
	En los apartados que siguen pondremos la atención en aquellos aspectos de la implementación que pueden ser relevantes a causa de su complejidad o particularidad. En estos se describen los inconvenientes que presentan y la forma en que fueron resueltos.
\bigskip



% DISEÑO - Cliente
\subsection{Cliente}
	
	Se ha decidido que la clase Cliente sea un Thread de manera de dejar la posibilidad de, en un futuro, agregar un terminal que permita al usuario del lado cliente interactuar con esta, ya sea, viendo el estado del procesamiento, detener el procesamiento manualmente, entre otras acciones posibles.
	\bigskip	



% DISEÑO - Servidor
\subsection{Servidor}

	\bigskip




% ESQUEMA DEL DISEÑO
\section{Esquema del diseño}

	A continuación, en la \textit{Figura 2}, se ilustra el diagrama de clases principal correspondiente al lado Servidor, donde se representa la relación entre las entidades que participan en dicha parte del sistema. 

%\newpage
% Figura 2
%\begin{figure}[h]
%	\centering
%	\includegraphics[width=0.968976\textwidth]{images/diagrama02.png}
%	\medskip
%	\caption{Diagrama de flujo representativo de la solución.}
%\end{figure}


	En la \textit{Figura 3}, se ilustra el diagrama de clases principal correspondiente al lado Cliente, mostrando aquí también la relación entre entidades que partician en esta otra parte del sistema.

%\newpage
% Figura 3
%\begin{figure}[h]
%	\centering
%	\includegraphics[width=0.968976\textwidth]{images/diagrama02.png}
%	\medskip
%	\caption{Diagrama de flujo representativo de la solución.}
%\end{figure}
\bigskip\bigskip




% REFERENCIAS
\begin{thebibliography}{99}

	\bibitem{ASCII} Código ASCII, \url{http://en.wikipedia.org/wiki/ASCII}
	\bibitem{FB} Ataque por Fuerza Bruta (Brute-force attack), \url{http://en.wikipedia.orig/wiki/Brute-force_attack}
	\end{thebibliography}

\newpage


\end{document}
